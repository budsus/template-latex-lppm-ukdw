\documentclass[12pt,a4paper]{report}
% TEMPLATE PROPOSAL PENELITIAN LPPM
% Dibuat oleh Budi Susanto (budsus@ti.ukdw.ac.id)
% Last Update: 16 Februari 2026

% ======================================================
% PENGATURAN HALAMAN
% ======================================================
\usepackage[a4paper,
left=3cm,
right=3cm,
top=3cm,
bottom=3cm]{geometry}
\usepackage[indonesian]{babel}
\usepackage{pdflscape}
\usepackage{pdfpages}
\usepackage{rotating}
\usepackage{setspace}
\usepackage{graphicx}
\usepackage{titlesec}
\usepackage{tocloft}
\usepackage{indentfirst}
\usepackage{enumitem}
\usepackage{booktabs}
\usepackage{array}
\usepackage{longtable}
\usepackage{multirow}
\usepackage[table,xcdraw]{xcolor}
\usepackage{caption}
\usepackage{float}
\usepackage{etoolbox}
\usepackage{newtxtext,newtxmath}
\usepackage[hidelinks]{hyperref}
\usepackage{natbib}
\bibliographystyle{apalike}

\onehalfspacing
\setlength{\parindent}{1.25cm}

% FORMAT BAB
% ===== CHAPTER =====
\titleformat{\chapter}[display]
	{\centering\bfseries\fontsize{16}{18}\selectfont}
	{BAB \thechapter}
	{0pt}                       % jarak label ke judul
	{\vspace{1ex}}              % sebelum judul

\titlespacing*{\chapter}{0pt}{0pt}{16pt}

% ===== SECTION =====
\titleformat{\section}
	{\normalfont\bfseries\fontsize{14}{16}\selectfont}
	{\thesection}
	{0.8em}
	{}

% ===== SUBSECTION =====
\titleformat{\subsection}
	{\normalfont\bfseries\normalsize}
	{\thesubsection}
	{0.7em}
	{}

% ===== DAFTAR ISI ======
\renewcommand{\contentsname}{DAFTAR ISI}
\renewcommand{\cfttoctitlefont}{\hfill\Large\bfseries}
\renewcommand{\cftaftertoctitle}{\hfill\mbox{}}
\addto\captionsindonesian{
	\renewcommand{\contentsname}{DAFTAR ISI}
}

\setlist{nosep}

\begin{document}
	
	% ===============================
	% SAMPUL
	% ===============================
	\begin{titlepage}
		\begin{flushright}
			\fbox{\small\textbf{[DIISI JENIS SKEMA PENELITIAN]}}
		\end{flushright}
		\centering
		\vspace{1cm}
		{\large USULAN PENELITIAN}\\[1cm]
		
		\includegraphics[width=4cm]{./fig/logo_ukdw.png}\\[1.5cm]
		
		{\large\bfseries [JUDUL PENELITIAN]}\\[2cm]
		
		{\large\bfseries TIM PENGUSUL}\\[0.5cm]
		Nama lengkap ketua (Ketua)\\
		Nama lengkap Anggota 1  \\
		Nama lengkap Anggota 2 \\
		Nama lengkap Anggota 3 \\[3cm]
		
		{\large FAKULTAS [NAMA FAKULTAS]}\\
		{\large UNIVERSITAS KRISTEN DUTA WACANA}\\[0.5cm]
		
		Yogyakarta\\
		Bulan Tahun
	\end{titlepage}
	
	\clearpage
	\pagenumbering{roman}
	
	%	% ======================================================
% HALAMAN PENGESAHAN
% ======================================================
\chapter*{HALAMAN PENGESAHAN}
\addcontentsline{toc}{chapter}{HALAMAN PENGESAHAN}

\begingroup
\fontsize{9}{9}
\begin{longtable}[c]{p{0.3\textwidth}cp{0.65\textwidth}}
	\textbf{Judul Penelitian} & : & JUDUL PENELITIAN \\
	\endfirsthead
	%
	\endhead
	%
	\textbf{Ketua Peneliti} &  &  \\
	\textbf{a. Nama Lengkap} & : &  \\
	\textbf{b. Program Studi} & : &  \\
	\textbf{c. Nomor HP} & : &  \\
	\textbf{d. Email} & : &  \\
	\textbf{Anggota Peneliti} &  &  \\
	\textbf{a. Nama Lengkap} & : &  \\
	\textbf{b. Nama Lengkap} & : &  \\
	\textbf{b. Nama Lengkap} & : &  \\
	\textbf{Mahasiswa} &  &  \\
	\textbf{a. Nama Lengkap/NIM} & : &  \\
	\textbf{b. Nama Lengkap/NIM} & : &  \\
	\textbf{Biaya Penelitian} & \multicolumn{1}{l}{} &  \\
	\textbf{a. Diusulkan ke LPPM} & : &  \\
	\textbf{b. Diusulkan ke Fakultas/Prodi} & : & - \\
	\textbf{c. Diusulkan institusi lain} & : & - \\
	\textbf{d. Dana Pribadi} & : & - \\
\end{longtable}
\endgroup
\noindent
Yogyakarta, Tgl Bulan Tahun
\begingroup
\fontsize{9}{9}	
\begin{longtable}[c]{p{0.5\textwidth}p{0.5\textwidth}}
	Mengetahui, & Ketua Peneliti, \\
	Dekan & \\
	\\
	\\
	(NAMA DEKAN FAKULTAS) & (NAMA KETUA) \\
	NIK:   & NIK:   \\
\end{longtable}
\endgroup	
\begin{center}
	Menyetujui,\\
	Ketua LPPM\\[1.5cm]
	(Dr. Freddy Marihot Rotua Nainggolan, S.T., M.T., IAI.)\\
	NIK: 154E399
\end{center}
	\addcontentsline{toc}{chapter}{HALAMAN PENGESAHAN}
	\includepdf[pages=-, fitpaper=true]{./pdf/HalSah.pdf}
	
	\tableofcontents
	
	\clearpage
	\addcontentsline{toc}{chapter}{RINGKASAN}
	\include{ringkasan}
	
	% ===============================
	% PANGGIL FILE BAB
	% ===============================
	\clearpage
	\pagenumbering{arabic}
	
	\chapter{PENDAHULUAN}


\section{Latar Belakang}
Bagian ini menjelaskan alasan mengapa penelitian perlu dilakukan. Isi utamanya meliputi:
\begin{itemize}
	\item Gambaran umum fenomena atau konteks penelitian.
	\item Data, fakta, atau tren yang menunjukkan adanya masalah atau kesenjangan.
	\item Tinjauan singkat penelitian sebelumnya atau kondisi teoretis yang relevan.
	\item Penjelasan mengenai gap penelitian atau persoalan yang belum terjawab.
	\item Argumen yang mengarah pada pentingnya penelitian dilakukan.
\end{itemize}
\bigbreak
\noindent
Tujuannya adalah membawa pembaca dari konteks umum menuju masalah spesifik yang akan diteliti.

\section{Rumusan Masalah}
Bagian ini memuat pernyataan masalah inti yang menjadi fokus penelitian. Isinya:
\begin{itemize}
	\item Ringkasan dari masalah utama yang muncul dalam latar belakang.
	\item Dinyatakan secara jelas, ringkas, dan terfokus.
	\item Biasanya berbentuk pernyataan, bukan pertanyaan.
\end{itemize}
\bigbreak
\noindent
Fungsinya adalah menunjukkan apa masalah pokok yang hendak diselesaikan penelitian.

\section{Tujuan Penelitian}
Bagian ini menjelaskan apa yang ingin dicapai melalui penelitian. Isinya:
\begin{itemize}
	\item Tujuan umum penelitian.
	\item Jika perlu, tujuan khusus yang lebih rinci.
	\item Harus selaras langsung dengan rumusan masalah.
\end{itemize}
\bigbreak
\noindent
Tujuan penelitian biasanya ditulis dengan kata kerja seperti:
\begin{itemize}
	\item Mengidentifikasi
	\item Menganalisis
	\item Menguji
	\item Mengembangkan
	\item Mengevaluasi
\end{itemize}

\section{Manfaat Penelitian}
Bagian ini menjelaskan kontribusi atau kegunaan hasil penelitian. Umumnya dibagi menjadi:
\begin{itemize}
	\item Manfaat teoretis: Kontribusi terhadap pengembangan ilmu, teori, atau konsep.
	\item Manfaat praktis: Kegunaan bagi praktisi, pembuat kebijakan, institusi, atau masyarakat.
	\item (Opsional) Manfaat metodologis atau kebijakan: Kontribusi terhadap metode penelitian atau rekomendasi kebijakan.
\end{itemize}
\bigbreak
\noindent
Fokusnya adalah menjawab: siapa yang diuntungkan dan bagaimana manfaatnya.

\section{Pertanyaan Penelitian}
Bagian ini berisi pertanyaan-pertanyaan spesifik yang akan dijawab melalui penelitian. Isinya:
\begin{itemize}
	\item Pertanyaan utama yang langsung berkaitan dengan rumusan masalah.
	\item Beberapa pertanyaan turunan yang lebih operasional.
	\item Disusun secara logis dan sistematis.
\end{itemize}
\bigbreak
\noindent
Pertanyaan penelitian berfungsi sebagai:
\begin{itemize}
	\item Panduan pengumpulan data.
	\item Dasar analisis.
	\item Pengarah keseluruhan proses penelitian.
\end{itemize}

	\chapter{LANDASAN TEORI dan TINJAUAN PUSTAKA}


\section{Landasan Teori}
Bagian Landasan Teori berisi konsep, teori, atau kerangka pemikiran utama yang menjadi dasar analisis penelitian. Fokusnya adalah teori inti yang menjelaskan variabel atau fenomena yang diteliti.
\bigbreak
\noindent
Isi yang perlu dituliskan:
\begin{enumerate}
	\item Definisi konsep utama dalam penelitian.
	\item Teori-teori yang relevan dari para ahli.
	\item Penjelasan hubungan antar konsep atau variabel.
	\item Argumentasi teoretis yang mendukung hipotesis atau model penelitian.
\end{enumerate}
\bigbreak
\noindent
Bagian ini bersifat konseptual dan teoritis, bukan sekadar ringkasan penelitian sebelumnya.
\bigbreak
\noindent
Contoh penulisan dengan sitasi
\begin{itemize}
	\item Sitasi naratif (nama penulis menjadi bagian kalimat):
	Menurut \cite{bourdieu1986}, kapital budaya berperan dalam membentuk posisi sosial individu melalui struktur habitus dan praktik sosial.
	
	\item Sitasi dalam tanda kurung:
	Literasi digital dipahami sebagai kemampuan untuk mengakses, mengevaluasi, dan memproduksi informasi secara kritis dalam lingkungan digital \citep{unesco2018}.
	
	\item Menyebut tahun saja:
	Konsep literasi AI mulai banyak dibahas dalam kerangka kompetensi masa depan \citeyearpar{unesco2023}.
	
	\item Sitasi dengan nomor halaman:
	Habitus dipahami sebagai “sistem disposisi yang terstruktur dan menstrukturkan praktik sosial” \citep[p. 72]{bourdieu1990}.
\end{itemize}

\section{Tinjauan Pustaka}
Bagian Tinjauan Pustaka berisi pembahasan penelitian-penelitian sebelumnya yang relevan dengan topik. Fokusnya adalah temuan empiris, metode, dan kesenjangan penelitian.
\bigbreak
\noindent
Isi yang perlu dituliskan:
\begin{enumerate}
	\item Ringkasan penelitian terdahulu yang relevan.
	\item Perbandingan pendekatan, metode, atau hasil penelitian.
	\item Identifikasi kesamaan dan perbedaan antar studi.
	\item Penjelasan \textit{research gap} yang menjadi dasar penelitian Anda.
\end{enumerate}
\bigbreak
\noindent
Bagian ini bersifat empiris dan komparatif, bukan teoritis murni.
\bigbreak
\noindent
Contoh penulisan dengan sitasi
\begin{itemize}
	\item Sitasi naratif:
	\cite{livingstone2014} menunjukkan bahwa literasi digital tidak hanya berkaitan dengan keterampilan teknis, tetapi juga kompetensi kritis dalam memahami struktur media.
	
	\item Sitasi dalam tanda kurung:
	Penelitian tentang literasi AI menunjukkan bahwa pemahaman terhadap bias algoritmik masih rendah pada sebagian besar pengguna \citep{long2021}.
	
	\item Menyebut tahun saja:
	Studi mengenai kompetensi digital generasi muda berkembang pesat setelah publikasi kerangka DigComp \citeyearpar{carretero2017}.
	
	\item Sitasi dengan halaman:
	Penelitian tersebut menegaskan bahwa “literasi digital mencakup dimensi teknis, kognitif, dan sosial” \citep[p. 45]{ng2012}.
\end{itemize}

	\chapter{METODOLOGI PENELITIAN}

\section{\textit{Roadmap} Penelitian}
Bagian Roadmap Penelitian menjelaskan peta besar arah penelitian dalam jangka menengah atau panjang. Roadmap menunjukkan kesinambungan topik, target luaran, serta posisi penelitian yang sedang diajukan dalam keseluruhan rencana penelitian.
\bigbreak
\noindent
Isi yang perlu dituliskan:
\begin{enumerate}
	\item Visi atau arah besar penelitian
	\begin{itemize}
		\item Tema utama atau fokus bidang penelitian dalam beberapa tahun ke depan.
		\item Permasalahan besar yang ingin diselesaikan.
	\end{itemize}
	
	\item Tahapan perkembangan penelitian
	\begin{itemize}
		\item Pembagian penelitian dalam beberapa fase (misalnya: eksplorasi, pengembangan model, uji intervensi, implementasi).
		\item Biasanya disusun per tahun atau per periode tertentu.
	\end{itemize}
	
	\item Keterkaitan antar penelitian
	\begin{itemize}
		\item Hubungan antara penelitian sebelumnya, penelitian yang sedang diajukan, dan rencana penelitian berikutnya.
		\item Menunjukkan kesinambungan topik.
	\end{itemize}
	
	\item Target luaran setiap tahap
	\begin{itemize}
		\item Publikasi ilmiah
		\item Model atau kerangka konseptual
		\item Modul pelatihan
		\item Produk atau rekomendasi kebijakan
	\end{itemize}
\end{enumerate}
\bigbreak
\noindent
Tujuan utama roadmap:
\begin{itemize}
	\item Menunjukkan bahwa penelitian tidak berdiri sendiri, tetapi merupakan bagian dari rencana strategis.
	\item Memberikan gambaran arah kontribusi jangka panjang.
\end{itemize}
\bigbreak
\noindent
Contoh narasi singkat:

\noindent
Penelitian ini merupakan bagian dari \textit{roadmap} penelitian literasi budaya digital yang dirancang selama tiga tahun. Tahun pertama difokuskan pada pengembangan model konseptual berbasis studi literatur dan survei. Tahun kedua diarahkan pada uji intervensi melalui pendekatan kuasi-eksperimental. Tahun ketiga berfokus pada implementasi model dalam skala yang lebih luas serta penyusunan rekomendasi kebijakan.

\begin{figure}[H]
	\centerline{\includegraphics[width=\columnwidth]{./fig/roadmap.png}}
	\caption{Peta Jalan Penelitian}
	\label{fig1}
\end{figure}

\section{Tahapan Penelitian}
Bagian Tahapan Penelitian menjelaskan langkah-langkah operasional yang dilakukan dalam penelitian yang sedang diajukan. Fokusnya adalah proses kerja penelitian secara sistematis.

\noindent
Isi yang perlu dituliskan:
\begin{enumerate}
	\item Tahap persiapan
	\begin{itemize}
		\item Studi literatur
		\item Penyusunan instrumen
		\item Validasi instrumen
	\end{itemize}
	
	\item Tahap pengumpulan data
	\begin{itemize}
		\item Penentuan sampel
		\item Pelaksanaan survei, wawancara, eksperimen, atau observasi
	\end{itemize}
	
	\item Tahap analisis data
	\begin{itemize}
		\item Teknik analisis yang digunakan (misalnya SEM-PLS, regresi, analisis tematik, dll.)
	\end{itemize}
	
	\item Tahap interpretasi dan penyusunan luaran
	\begin{itemize}
		\item Penarikan kesimpulan
		\item Penyusunan laporan, artikel, atau produk penelitian
	\end{itemize}
	
	\item (Opsional) Tahap diseminasi
	\begin{itemize}
		\item Seminar
		\item Publikasi
		\item Penyusunan rekomendasi
	\end{itemize}
\end{enumerate}
\bigbreak
\begin{figure}[H]
	\centerline{\includegraphics[width=0.6\columnwidth]{./fig/tahapan.png}}
	\caption{Tahapan Penelitian}
	\label{fig2}
\end{figure}
\bigbreak
\noindent
Tujuan utama tahapan penelitian:
\begin{itemize}
	\item Menunjukkan alur kerja penelitian secara jelas dan logis.
	\item Memastikan penelitian dapat dilakukan secara sistematis dan terukur.
\end{itemize}
\bigbreak
\noindent
Contoh narasi singkat:

\noindent
Penelitian ini dilaksanakan melalui empat tahap utama. Tahap pertama adalah persiapan, meliputi studi literatur dan penyusunan instrumen survei. Tahap kedua adalah pengumpulan data melalui survei kepada responden yang dipilih secara purposive. Tahap ketiga adalah analisis data menggunakan pendekatan SEM-PLS. Tahap keempat adalah interpretasi hasil dan penyusunan luaran penelitian berupa artikel ilmiah dan rekomendasi kebijakan.

	\chapter{BIAYA dan JADWAL PENELITIAN}

\section{Anggaran Biaya}
\begin{longtable}[c]{|cl|r|c|}
	\hline
	\rowcolor[HTML]{EFEFEF} 
	\multicolumn{1}{|c|}{\cellcolor[HTML]{EFEFEF}\textbf{No}} & \multicolumn{1}{c|}{\cellcolor[HTML]{EFEFEF}\textbf{Jenis Pengeluaran}} & \multicolumn{1}{c|}{\cellcolor[HTML]{EFEFEF}\textbf{Biaya yang diusulkan (Rp)}} & \textbf{\%} \\ \hline
	\endfirsthead
	%
	\endhead
	%
	\multicolumn{1}{|c|}{1} & Honorarium (Gaji dan upah maks. 30\%) & 0 & 0\% \\ \hline
	\multicolumn{1}{|c|}{2} & Bahan habis pakai dan peralatan (40-60\%) & 0 & 0\% \\ \hline
	\multicolumn{1}{|c|}{3} & Perjalanan (maks. 15\%) & 00 & 0\% \\ \hline
	\multicolumn{1}{|c|}{4} & \begin{tabular}[c]{@{}l@{}}Lain-lain (publikasi, seminar, laporan,\\ lainnya sebutkan) (10- 15\%)\end{tabular} & 0 & 0\% \\ \hline
	\multicolumn{2}{|r|}{\textbf{Jumlah}} & \textbf{0} &  \\ \hline
\end{longtable}

\section{Jadwal Penelitian}
\begin{longtable}[c]{|l|l|l|l|l|l|l|l|l|l|}
	\hline
	\rowcolor[HTML]{C0C0C0} 
	\multicolumn{1}{|c|}{\cellcolor[HTML]{C0C0C0}} & \multicolumn{9}{c|}{\cellcolor[HTML]{C0C0C0}\textbf{Bulan ke-}} \\ \cline{2-10} 
	\rowcolor[HTML]{C0C0C0} 
	\multicolumn{1}{|c|}{\multirow{-2}{*}{\cellcolor[HTML]{C0C0C0}\textbf{Kegiatan}}} & \multicolumn{1}{c|}{\cellcolor[HTML]{C0C0C0}\textbf{1}} & \multicolumn{1}{c|}{\cellcolor[HTML]{C0C0C0}\textbf{2}} & \multicolumn{1}{c|}{\cellcolor[HTML]{C0C0C0}\textbf{3}} & \multicolumn{1}{c|}{\cellcolor[HTML]{C0C0C0}\textbf{4}} & \multicolumn{1}{c|}{\cellcolor[HTML]{C0C0C0}\textbf{5}} & \multicolumn{1}{c|}{\cellcolor[HTML]{C0C0C0}\textbf{6}} & \multicolumn{1}{c|}{\cellcolor[HTML]{C0C0C0}\textbf{7}} & \multicolumn{1}{c|}{\cellcolor[HTML]{C0C0C0}\textbf{8}} & \multicolumn{1}{c|}{\cellcolor[HTML]{C0C0C0}\textbf{9}} \\ \hline
	\endfirsthead
	%
	\endhead
	%
	\rowcolor[HTML]{EFEFEF} 
	Tahap 1 & \checkmark & \checkmark &  &  &  &  &  &  &  \\ \hline
	\rowcolor[HTML]{EFEFEF} 
	Tahap 2 & \checkmark & \checkmark &  &  &  &  &  &  &  \\ \hline
	\rowcolor[HTML]{EFEFEF} 
	Tahap 3 &  & \checkmark &  &  &  &  &  &  &  \\ \hline
	Tahap 4 &  &  & \checkmark & \checkmark &  &  &  &  &  \\ \hline
	Tahap 5 &  &  & \checkmark & \checkmark &  &  &  &  &  \\ \hline
	Tahap 6 &  &  &  & \checkmark &  &  &  &  &  \\ \hline
	Tahap 7 &  &  &  & \checkmark &  &  &  &  &  \\ \hline
	\rowcolor[HTML]{EFEFEF} 
	Tahap 8 &  &  &  &  & \checkmark & \checkmark &  &  &  \\ \hline
	\rowcolor[HTML]{EFEFEF} 
	Tahap 9 &  &  &  &  &  & \checkmark & \checkmark &  &  \\ \hline
	\rowcolor[HTML]{EFEFEF} 
	Tahap 10 &  &  &  &  &  & \checkmark & \checkmark &  &  \\ \hline
	Tahap 11 &  &  &  &  &  &  & \checkmark & \checkmark &  \\ \hline
	Tahap 12 &  &  &  &  &  &  &  & \checkmark & \checkmark \\ \hline
	Tahap 13 &  &  &  &  &  &  &  & \checkmark & \checkmark \\ \hline
\end{longtable}
	
	% ===============================
	% DAFTAR PUSTAKA
	% ===============================
	\cleardoublepage
	\phantomsection
	\addcontentsline{toc}{chapter}{DAFTAR PUSTAKA}
	\renewcommand{\bibname}{DAFTAR PUSTAKA}
	%\bibliographystyle{apalike} % atau style lain
	\bibliography{referensi}
	
	% ===============================
	% LAMPIRAN
	% ===============================
	\appendix
	\clearpage
	\thispagestyle{empty}
	\addcontentsline{toc}{chapter}{LAMPIRAN}
	
	\null
	\vfill
	\begin{center}
		{\Large\bfseries LAMPIRAN}
	\end{center}
	\vfill
	\null
	\clearpage
	\begin{sidewaystable}
	\section*{LAMPIRAN 1. Justifikasi Anggaran}
	\addcontentsline{toc}{section}{LAMPIRAN 1. Justifikasi Anggaran}
	
	\centering
	\begin{longtable}[c]{|llrrr|}
		\hline
		\multicolumn{5}{|l|}{\cellcolor[HTML]{D9D9D9}1. Honor} \\ \hline
		\endfirsthead
		%
		\endhead
		%
		\multicolumn{1}{|l|}{\textbf{Honor}} & \multicolumn{1}{l|}{\textbf{Honor/Jam (Rp)}} & \multicolumn{1}{l|}{\textbf{Waktu (jam/minggu)}} & \multicolumn{1}{l|}{\textbf{Minggu}} & \multicolumn{1}{l|}{\textbf{Anggaran (Rp)}} \\ \hline
		\multicolumn{1}{|l|}{Ketua (1)} & \multicolumn{1}{r|}{} & \multicolumn{1}{r|}{} & \multicolumn{1}{r|}{} &  \\ \hline
		\multicolumn{1}{|l|}{Anggota (1)} & \multicolumn{1}{r|}{} & \multicolumn{1}{r|}{} & \multicolumn{1}{r|}{} &  \\ \hline
		\multicolumn{1}{|l|}{Anggota (2)} & \multicolumn{1}{r|}{} & \multicolumn{1}{r|}{} & \multicolumn{1}{r|}{} &  \\ \hline
		\multicolumn{1}{|l|}{Anggota (3)} & \multicolumn{1}{r|}{} & \multicolumn{1}{r|}{} & \multicolumn{1}{r|}{} &  \\ \hline
		\multicolumn{4}{|l|}{\cellcolor[HTML]{EFEFEF}Sub Total Honor} &  \\ \hline
		\multicolumn{5}{|l|}{\cellcolor[HTML]{D9D9D9}2. Peralatan Penunjang} \\ \hline
		\multicolumn{1}{|l|}{\textbf{Material}} & \multicolumn{1}{l|}{\textbf{Justifikasi Pemakaian}} & \multicolumn{1}{l|}{\textbf{Kuantitas}} & \multicolumn{1}{l|}{\textbf{Harga Satuan (Rp)}} & \multicolumn{1}{l|}{\textbf{Anggaran (Rp)}} \\ \hline
		\multicolumn{1}{|l|}{} & \multicolumn{1}{l|}{} & \multicolumn{1}{r|}{} & \multicolumn{1}{r|}{} &  \\ \hline
		\multicolumn{1}{|l|}{} & \multicolumn{1}{l|}{} & \multicolumn{1}{l|}{} & \multicolumn{1}{l|}{} &  \\ \hline
		\multicolumn{4}{|l|}{\cellcolor[HTML]{EFEFEF}Sub Total Peralatan Penunjang} &  \\ \hline
		\multicolumn{5}{|l|}{\cellcolor[HTML]{D9D9D9}3. Bahan Habis Pakai} \\ \hline
		\multicolumn{1}{|l|}{\textbf{Material}} & \multicolumn{1}{l|}{\textbf{Justifikasi Pemakaian}} & \multicolumn{1}{l|}{\textbf{Kuantitas}} & \multicolumn{1}{l|}{\textbf{Harga Satuan (Rp)}} & \multicolumn{1}{l|}{\textbf{Anggaran (Rp)}} \\ \hline
		\multicolumn{1}{|l|}{} & \multicolumn{1}{l|}{} & \multicolumn{1}{r|}{} & \multicolumn{1}{r|}{} &  \\ \hline
		\multicolumn{1}{|l|}{} & \multicolumn{1}{l|}{} & \multicolumn{1}{l|}{} & \multicolumn{1}{l|}{} &  \\ \hline
		\multicolumn{4}{|l|}{\cellcolor[HTML]{EFEFEF}Sub Total Bahan Habis Pakai} &  \\ \hline
		\multicolumn{5}{|l|}{\cellcolor[HTML]{D9D9D9}4. Perjalanan} \\ \hline
		\multicolumn{1}{|l|}{\textbf{Material}} & \multicolumn{1}{l|}{\textbf{Justifikasi Pemakaian}} & \multicolumn{1}{l|}{\textbf{Kuantitas}} & \multicolumn{1}{l|}{\textbf{Harga Satuan (Rp)}} & \multicolumn{1}{l|}{\textbf{Anggaran (Rp)}} \\ \hline
		\multicolumn{1}{|l|}{} & \multicolumn{1}{l|}{} & \multicolumn{1}{r|}{} & \multicolumn{1}{r|}{} &  \\ \hline
		\multicolumn{1}{|l|}{} & \multicolumn{1}{l|}{} & \multicolumn{1}{l|}{} & \multicolumn{1}{l|}{} &  \\ \hline
		\multicolumn{4}{|l|}{\cellcolor[HTML]{EFEFEF}Sub Total Perjalanan} &  \\ \hline
		\multicolumn{5}{|l|}{\cellcolor[HTML]{D9D9D9}5. Lain lain} \\ \hline
		\multicolumn{1}{|l|}{\textbf{Material}} & \multicolumn{1}{l|}{\textbf{Justifikasi Pemakaian}} & \multicolumn{1}{l|}{\textbf{Kuantitas}} & \multicolumn{1}{l|}{\textbf{Harga Satuan (Rp)}} & \multicolumn{1}{l|}{\textbf{Anggaran (Rp)}} \\ \hline
		\multicolumn{1}{|l|}{} & \multicolumn{1}{l|}{} & \multicolumn{1}{r|}{} & \multicolumn{1}{r|}{} &  \\ \hline
		\multicolumn{1}{|l|}{} & \multicolumn{1}{l|}{} & \multicolumn{1}{r|}{} & \multicolumn{1}{r|}{} &  \\ \hline
		\multicolumn{4}{|l|}{\cellcolor[HTML]{EFEFEF}Sub Total Lain-lain} &  \\ \hline
		\multicolumn{4}{|l|}{\cellcolor[HTML]{000000}\textbf{Total Anggaran yang Diperlukan (Rp)}} &  \\ \hline
	\end{longtable}
\end{sidewaystable}

	\clearpage	
	\null
	\vfill
	\begin{center}
		\textbf{LAMPIRAN 2. BIODATA KETUA dan ANGGOTA}
	\end{center}
	\vfill
	\null
	\addcontentsline{toc}{section}{LAMPIRAN 2. Biodata Ketua dan Anggota}
	\clearpage
	\includepdf[pages=-, fitpaper=true]{./pdf/TemplateBIODATA.pdf}
	
	\addcontentsline{toc}{section}{LAMPIRAN 3. Surat Pernyataan Ketua Peneliti}
	\includepdf[
	pages=-,
	pagecommand={
		\thispagestyle{plain}
		\section*{LAMPIRAN 3. Surat Pernyataan Ketua Peneliti/Pelaksana}
	}
	]{./pdf/SuratPernyataanKetua.pdf}

\end{document}