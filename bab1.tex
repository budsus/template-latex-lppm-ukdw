\chapter{PENDAHULUAN}


\section{Latar Belakang}
Bagian ini menjelaskan alasan mengapa penelitian perlu dilakukan. Isi utamanya meliputi:
\begin{itemize}
	\item Gambaran umum fenomena atau konteks penelitian.
	\item Data, fakta, atau tren yang menunjukkan adanya masalah atau kesenjangan.
	\item Tinjauan singkat penelitian sebelumnya atau kondisi teoretis yang relevan.
	\item Penjelasan mengenai gap penelitian atau persoalan yang belum terjawab.
	\item Argumen yang mengarah pada pentingnya penelitian dilakukan.
\end{itemize}
\bigbreak
\noindent
Tujuannya adalah membawa pembaca dari konteks umum menuju masalah spesifik yang akan diteliti.

\section{Rumusan Masalah}
Bagian ini memuat pernyataan masalah inti yang menjadi fokus penelitian. Isinya:
\begin{itemize}
	\item Ringkasan dari masalah utama yang muncul dalam latar belakang.
	\item Dinyatakan secara jelas, ringkas, dan terfokus.
	\item Biasanya berbentuk pernyataan, bukan pertanyaan.
\end{itemize}
\bigbreak
\noindent
Fungsinya adalah menunjukkan apa masalah pokok yang hendak diselesaikan penelitian.

\section{Tujuan Penelitian}
Bagian ini menjelaskan apa yang ingin dicapai melalui penelitian. Isinya:
\begin{itemize}
	\item Tujuan umum penelitian.
	\item Jika perlu, tujuan khusus yang lebih rinci.
	\item Harus selaras langsung dengan rumusan masalah.
\end{itemize}
\bigbreak
\noindent
Tujuan penelitian biasanya ditulis dengan kata kerja seperti:
\begin{itemize}
	\item Mengidentifikasi
	\item Menganalisis
	\item Menguji
	\item Mengembangkan
	\item Mengevaluasi
\end{itemize}

\section{Manfaat Penelitian}
Bagian ini menjelaskan kontribusi atau kegunaan hasil penelitian. Umumnya dibagi menjadi:
\begin{itemize}
	\item Manfaat teoretis: Kontribusi terhadap pengembangan ilmu, teori, atau konsep.
	\item Manfaat praktis: Kegunaan bagi praktisi, pembuat kebijakan, institusi, atau masyarakat.
	\item (Opsional) Manfaat metodologis atau kebijakan: Kontribusi terhadap metode penelitian atau rekomendasi kebijakan.
\end{itemize}
\bigbreak
\noindent
Fokusnya adalah menjawab: siapa yang diuntungkan dan bagaimana manfaatnya.

\section{Pertanyaan Penelitian}
Bagian ini berisi pertanyaan-pertanyaan spesifik yang akan dijawab melalui penelitian. Isinya:
\begin{itemize}
	\item Pertanyaan utama yang langsung berkaitan dengan rumusan masalah.
	\item Beberapa pertanyaan turunan yang lebih operasional.
	\item Disusun secara logis dan sistematis.
\end{itemize}
\bigbreak
\noindent
Pertanyaan penelitian berfungsi sebagai:
\begin{itemize}
	\item Panduan pengumpulan data.
	\item Dasar analisis.
	\item Pengarah keseluruhan proses penelitian.
\end{itemize}
