\chapter{METODOLOGI PENELITIAN}

\section{\textit{Roadmap} Penelitian}
Bagian Roadmap Penelitian menjelaskan peta besar arah penelitian dalam jangka menengah atau panjang. Roadmap menunjukkan kesinambungan topik, target luaran, serta posisi penelitian yang sedang diajukan dalam keseluruhan rencana penelitian.
\bigbreak
\noindent
Isi yang perlu dituliskan:
\begin{enumerate}
	\item Visi atau arah besar penelitian
	\begin{itemize}
		\item Tema utama atau fokus bidang penelitian dalam beberapa tahun ke depan.
		\item Permasalahan besar yang ingin diselesaikan.
	\end{itemize}
	
	\item Tahapan perkembangan penelitian
	\begin{itemize}
		\item Pembagian penelitian dalam beberapa fase (misalnya: eksplorasi, pengembangan model, uji intervensi, implementasi).
		\item Biasanya disusun per tahun atau per periode tertentu.
	\end{itemize}
	
	\item Keterkaitan antar penelitian
	\begin{itemize}
		\item Hubungan antara penelitian sebelumnya, penelitian yang sedang diajukan, dan rencana penelitian berikutnya.
		\item Menunjukkan kesinambungan topik.
	\end{itemize}
	
	\item Target luaran setiap tahap
	\begin{itemize}
		\item Publikasi ilmiah
		\item Model atau kerangka konseptual
		\item Modul pelatihan
		\item Produk atau rekomendasi kebijakan
	\end{itemize}
\end{enumerate}
\bigbreak
\noindent
Tujuan utama roadmap:
\begin{itemize}
	\item Menunjukkan bahwa penelitian tidak berdiri sendiri, tetapi merupakan bagian dari rencana strategis.
	\item Memberikan gambaran arah kontribusi jangka panjang.
\end{itemize}
\bigbreak
\noindent
Contoh narasi singkat:

\noindent
Penelitian ini merupakan bagian dari \textit{roadmap} penelitian literasi budaya digital yang dirancang selama tiga tahun. Tahun pertama difokuskan pada pengembangan model konseptual berbasis studi literatur dan survei. Tahun kedua diarahkan pada uji intervensi melalui pendekatan kuasi-eksperimental. Tahun ketiga berfokus pada implementasi model dalam skala yang lebih luas serta penyusunan rekomendasi kebijakan.

\begin{figure}[H]
	\centerline{\includegraphics[width=\columnwidth]{./fig/roadmap.png}}
	\caption{Peta Jalan Penelitian}
	\label{fig1}
\end{figure}

\section{Tahapan Penelitian}
Bagian Tahapan Penelitian menjelaskan langkah-langkah operasional yang dilakukan dalam penelitian yang sedang diajukan. Fokusnya adalah proses kerja penelitian secara sistematis.

\noindent
Isi yang perlu dituliskan:
\begin{enumerate}
	\item Tahap persiapan
	\begin{itemize}
		\item Studi literatur
		\item Penyusunan instrumen
		\item Validasi instrumen
	\end{itemize}
	
	\item Tahap pengumpulan data
	\begin{itemize}
		\item Penentuan sampel
		\item Pelaksanaan survei, wawancara, eksperimen, atau observasi
	\end{itemize}
	
	\item Tahap analisis data
	\begin{itemize}
		\item Teknik analisis yang digunakan (misalnya SEM-PLS, regresi, analisis tematik, dll.)
	\end{itemize}
	
	\item Tahap interpretasi dan penyusunan luaran
	\begin{itemize}
		\item Penarikan kesimpulan
		\item Penyusunan laporan, artikel, atau produk penelitian
	\end{itemize}
	
	\item (Opsional) Tahap diseminasi
	\begin{itemize}
		\item Seminar
		\item Publikasi
		\item Penyusunan rekomendasi
	\end{itemize}
\end{enumerate}
\bigbreak
\begin{figure}[H]
	\centerline{\includegraphics[width=0.6\columnwidth]{./fig/tahapan.png}}
	\caption{Tahapan Penelitian}
	\label{fig2}
\end{figure}
\bigbreak
\noindent
Tujuan utama tahapan penelitian:
\begin{itemize}
	\item Menunjukkan alur kerja penelitian secara jelas dan logis.
	\item Memastikan penelitian dapat dilakukan secara sistematis dan terukur.
\end{itemize}
\bigbreak
\noindent
Contoh narasi singkat:

\noindent
Penelitian ini dilaksanakan melalui empat tahap utama. Tahap pertama adalah persiapan, meliputi studi literatur dan penyusunan instrumen survei. Tahap kedua adalah pengumpulan data melalui survei kepada responden yang dipilih secara purposive. Tahap ketiga adalah analisis data menggunakan pendekatan SEM-PLS. Tahap keempat adalah interpretasi hasil dan penyusunan luaran penelitian berupa artikel ilmiah dan rekomendasi kebijakan.
