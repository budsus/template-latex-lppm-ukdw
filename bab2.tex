\chapter{LANDASAN TEORI dan TINJAUAN PUSTAKA}


\section{Landasan Teori}
Bagian Landasan Teori berisi konsep, teori, atau kerangka pemikiran utama yang menjadi dasar analisis penelitian. Fokusnya adalah teori inti yang menjelaskan variabel atau fenomena yang diteliti.
\bigbreak
\noindent
Isi yang perlu dituliskan:
\begin{enumerate}
	\item Definisi konsep utama dalam penelitian.
	\item Teori-teori yang relevan dari para ahli.
	\item Penjelasan hubungan antar konsep atau variabel.
	\item Argumentasi teoretis yang mendukung hipotesis atau model penelitian.
\end{enumerate}
\bigbreak
\noindent
Bagian ini bersifat konseptual dan teoritis, bukan sekadar ringkasan penelitian sebelumnya.
\bigbreak
\noindent
Contoh penulisan dengan sitasi
\begin{itemize}
	\item Sitasi naratif (nama penulis menjadi bagian kalimat):
	Menurut \cite{bourdieu1986}, kapital budaya berperan dalam membentuk posisi sosial individu melalui struktur habitus dan praktik sosial.
	
	\item Sitasi dalam tanda kurung:
	Literasi digital dipahami sebagai kemampuan untuk mengakses, mengevaluasi, dan memproduksi informasi secara kritis dalam lingkungan digital \citep{unesco2018}.
	
	\item Menyebut tahun saja:
	Konsep literasi AI mulai banyak dibahas dalam kerangka kompetensi masa depan \citeyearpar{unesco2023}.
	
	\item Sitasi dengan nomor halaman:
	Habitus dipahami sebagai “sistem disposisi yang terstruktur dan menstrukturkan praktik sosial” \citep[p. 72]{bourdieu1990}.
\end{itemize}

\section{Tinjauan Pustaka}
Bagian Tinjauan Pustaka berisi pembahasan penelitian-penelitian sebelumnya yang relevan dengan topik. Fokusnya adalah temuan empiris, metode, dan kesenjangan penelitian.
\bigbreak
\noindent
Isi yang perlu dituliskan:
\begin{enumerate}
	\item Ringkasan penelitian terdahulu yang relevan.
	\item Perbandingan pendekatan, metode, atau hasil penelitian.
	\item Identifikasi kesamaan dan perbedaan antar studi.
	\item Penjelasan \textit{research gap} yang menjadi dasar penelitian Anda.
\end{enumerate}
\bigbreak
\noindent
Bagian ini bersifat empiris dan komparatif, bukan teoritis murni.
\bigbreak
\noindent
Contoh penulisan dengan sitasi
\begin{itemize}
	\item Sitasi naratif:
	\cite{livingstone2014} menunjukkan bahwa literasi digital tidak hanya berkaitan dengan keterampilan teknis, tetapi juga kompetensi kritis dalam memahami struktur media.
	
	\item Sitasi dalam tanda kurung:
	Penelitian tentang literasi AI menunjukkan bahwa pemahaman terhadap bias algoritmik masih rendah pada sebagian besar pengguna \citep{long2021}.
	
	\item Menyebut tahun saja:
	Studi mengenai kompetensi digital generasi muda berkembang pesat setelah publikasi kerangka DigComp \citeyearpar{carretero2017}.
	
	\item Sitasi dengan halaman:
	Penelitian tersebut menegaskan bahwa “literasi digital mencakup dimensi teknis, kognitif, dan sosial” \citep[p. 45]{ng2012}.
\end{itemize}
